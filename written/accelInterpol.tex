\documentclass{article}

\usepackage[utf8]{inputenc}
\usepackage{xspace}
\usepackage{tabu}
\usepackage[%
  hyperindex,%
  plainpages=false,%
  pdfusetitle]{hyperref}
\usepackage[all]{hypcap}
\usepackage{cite}
\usepackage{booktabs}
\usepackage{url}
\usepackage{listings}
\usepackage{enumitem}
\usepackage{amsthm}
\usepackage{amsmath}
\usepackage{tikz}
\usetikzlibrary{positioning,shapes.geometric, arrows,automata, decorations.pathreplacing, calc}
\usepackage{pgf}
\usepackage{slantsc}
\usepackage{geometry}
\usepackage{amssymb}
\usepackage{subcaption}
\usepackage{float}
\usepackage{pgf}
\usepackage{slashbox}
\usepackage{pgfgantt}
\usepackage{wrapfig}
\usepackage{pdflscape}
\usepackage{xcolor}
\usepackage{tcolorbox}% http://ctan.org/pkg/tcolorbox
\tcbuselibrary{skins,breakable}


\newtcbox{\mybox}{breakable, colframe=grey!50!black,colback=grey!40!white,
	boxrule=0.5pt,arc=4pt,boxsep=0pt,left=6pt,right=6pt,top=6pt,bottom=6pt}

\newcommand{\tf}{\ensuremath{\varphi}\xspace}
\newcommand{\ctf}{\ensuremath{\widehat{\varphi}}\xspace}
\newcommand{\invars}{\ensuremath{In}\xspace}
\newcommand{\outvars}{\ensuremath{Out}\xspace}
\newcommand{\auxvars}{\ensuremath{Aux}\xspace}

\newcommand{\WidestEntry}{$\psi_{global}$}%
\newcommand{\SetToWidest}[1]{\makebox[\widthof{\WidestEntry}]{$#1$}}%


\usepackage[%disable,%
  colorinlistoftodos,%
  color=cyan!50!white,%
  bordercolor=cyan!50!black]{todonotes}

%%%%%%%%%%%% Colors 
%% a somewhat friendly scheme for 5 different colors 
\definecolor{g1}		{RGB}{215,25,28} % a kind of red
\definecolor{g2}		{RGB}{253,174,97} % a kind of orange
\definecolor{g3}		{RGB}{255,255,191} % a kind of yellow
\definecolor{g4}		{RGB}{171,217,233} % a kind of light blue 
\definecolor{g5}		{RGB}{44,123,182} % a kind of dark blue 

\definecolor{gr1}		{RGB}{250, 250, 250}
\definecolor{gr2}		{RGB}{229, 229, 229} % some grey

% color of interpolants
\definecolor{grey}{RGB}{200,200,200}

%color for pictures
\colorlet{outlineblue}		{g5}
\colorlet{fillblue}			{g4}
\colorlet{darkback}			{gr2}
\colorlet{lightback}		{gr1}
\colorlet{stmtcolor}		{gr2} %default statement color
\colorlet{subgraphcolor}	{g3} %default statement color


%%%%%%%%%%%% Setup
\newtheorem{name}{Printed output}
\newtheorem{mydef}{Definition}

\hypersetup{
colorlinks=true,        % false: boxed links; true: colored links
linkcolor=g1,        % color of internal links
citecolor=g1,        % color of links to bibliography
filecolor=g1,        % color of file links
urlcolor=g1          % color of external links
}


\lstdefinestyle{boogie}{
  belowcaptionskip=1\baselineskip,
  breaklines=true,
  xleftmargin=\parindent,
  showstringspaces=false,
  basicstyle=\footnotesize\ttfamily,
  numbers=left,
  xleftmargin=.6cm
}

\lstset{escapechar=@,style=boogie}

%%%%%%%%%%%% Comments
\newif\iffinal
%\finaltrue % comment out to remove comments 
 
\iffinal
\newcommand\mycom[1]{}
\else
\newcommand\mycom[1]{#1}
\overfullrule=1mm
\fi
\setlength\parindent{0pt}

\newcommand{\jw}[1]{\mycom{\todo[color=blue!40,inline]{\small JW: #1}}}
\newcommand{\dd}[1]{\mycom{\todo[color=orange!40,inline]{\small DD: #1}}}
\newcommand{\ts}[1]{\mycom{\todo[color=green!40,inline]{\small TS: #1}}}


\newcommand{\all}[1]{\mycom{\todo[color=green!40,inline]{\small #1}}}
\newcommand{\meta}[1]{\mycom{\todo[color=blue!10,inline,caption={Beschreibung},nolist]{\setlist{nolistsep}\small #1}}}
\newcommand{\xxx}{\mycom{\stfootcol{Placeholder}{blue!20}\xspace}}
\newcommand{\cn}{\mycom{\stfootcol{Cite}{blue!20}\xspace}}


\begin{document}
	\newcommand{\HorizontalLine}{\rule{\linewidth}{0.3mm}}
	
		\begin{center}
		{\scshape\Large Master Project \par}
		\vspace{1.5cm}
		{\huge\bfseries Accelerated Interpolation \par}
		\vspace{1cm}
		{\large \scshape Jonas Werner\par}
		\vspace{0.5cm}
		{\today \vspace{2cm}} 
		
		\end{center}

\section{Introduction}
Assume we are given a program $P$ and a safety property and want to verify whether $P$ fulfils this property. $P$ consists of a set of program statements, like assignments: \texttt{x := 0}, or assumptions: \texttt{x != 42}, these program statements form a finite alphabet $\Sigma$. A program trace is a sequence of program statements, or a word over $\Sigma$. \\
A program can be represented as a directed labelled graph, called control-flow graph, with program locations as nodes and edges labelled with program statements, with a distinct initial location. The control-flow graph depicts all possible transitions from one program location to another. An error location is a program location, that violates the safety property. The goal is to check whether there is a program trace, called error trace, that starts in the initial node and ends in the error location, that is feasible. 

Trace abstraction aims at constructing automata \cite{10.1007/978-3-642-39799-8_2} from infeasible error traces. When the language recognized by the program's control-flow graph is a subset of these automata, it means, every possible error trace, and by that execution which ends in an error location, is infeasible. Proving that the program fulfills the safety property. If there is a single feasible error trace, then the program violates the safety property.\\
\par 
If an error trace is infeasible there is an infeasibility proof. Using this proof one can construct an automaton, which excludes the original error trace from the control-flow graph. However, this excludes only one error trace, making this approach not very efficient. To exclude more than one error trace, one can try to compute a generalization of the infeasibility proof. \par

A common strategy is \cite{10.1007/978-3-642-03237-0_7} to calculate a generalization using Craig interpolation \cite{craig_1957}, where an SMT-solver computes a sequence of interpolants from an infeasibility proof. \par 
But, it is not guaranteed that these interpolants are more general. This issue is most notably in program loops. Assume that the given program contains a loop with guard $x < 5000$, $x$ being an integer variable. It is possible that the computed interpolant sequence does not exclude every loop iteration, leading to the need of disproving every of the 5000 possible error traces individually. \\ \\
A solution for this problem is accelerating the loop, meaning computing its transitive closure, and calculating interpolants on that. \par This project aims at implementing exactly that. 
The goal is to combine interpolation and loop acceleration on the basis of the work of Hojjat et al \cite{10.1007/978-3-642-33386-6_16} in the software analysis framework \\ Ultimate \cite{Zitat02}. \par
The remainder of this proposal is structured as follows. Chapter 2 will give an overview of needed background information, like a more detailed look at trace abstraction, loop acceleration, and the combination of interpolation and acceleration. Chapter 3 will detail the approach this project will take to implement accelerated interpolation in Ultimate, and finally an outline of the project's deliverables and schedule.

\section{Background}
This project aims at combining loop acceleration and interpolant calculation, based on the findings of Hojjat et al. \cite{10.1007/978-3-642-33386-6_16}, however, instead of utilizing a CEGAR-scheme with predicate abstraction, it was be implemented as an automata-based CEGAR-scheme, called trace abstraction. \par
This section will introduce the basic ideas behind trace abstraction, loop acceleration, and finally accelerated interpolants.

\subsection{Programs}
Assume we are given a program and want to verify its safety by checking reachability of an error location. 
To apply interpolating trace abstraction one firstly needs to know what a program is \cite{DBLP:journals/corr/GreitschusDP17}. \\ 

In this paper, we consider a programming language consisting of the basic program statements assignment, assume, and sequential composition. Its syntax is denoted by the following grammar:
\begin{equation*}
	\texttt{s := assume bexpr | x := expr | s;s}
\end{equation*}
Where, given a finite set of program variables $Var$, \texttt{expr} is an expression over $Var$ and \texttt{bexpr} is a Boolean expression over $Var$. For brevity's sake we use \texttt{bexpr} instead of \texttt{assume bexpr}.
Using this programming language, it is possible to represent a program as follows.
\begin{mydef}
	A program $P$ over a given set of statements $Stmt$ can be represented as a labeled graph, called control-flow graph, $G_P = (Loc, \delta, \ell_{init}, \ell_{err})$, with $Loc$ being a finite set of program locations, a set of edges between two program locations labeled with a statement $\delta \subseteq Loc \times Stmt \times Loc$, an initial location $\ell_0 \in Loc$, and an error location $\ell_{err} \in Loc$.
\end{mydef}

Consider for example the following program $P_0$:
\begin{figure}[H]
	\begin{align*}
		&\texttt{1: int x := 0}; \\
		&\texttt{2: while x < 6:} \\
		&\texttt{3: \hspace*{2em} x := x + 2;} \\
		&\texttt{4: end while} \\
		&\texttt{5: assert x == 6;}
	\end{align*}
	\captionof{figure}{Example Program $P_0$.}
	\label{fig:square}
\end{figure}

$P_0$ is defined over the set of program statements: 
\begin{equation*}
	\Sigma = (\texttt{x := 0},\texttt{x < 6}, \texttt{x := x + 2}, \texttt{assume !x < 6}, \texttt{x == 6}, \texttt{x != 6})
\end{equation*} 
The program can be presented as control-flow graph $G_{P_0} = (Loc_P, \delta_P, \ell_{init}, \ell_{err})$, with 
\begin{itemize}
	\item $Loc_P = \{ \ell_1, \ell_2, \ell_3, \ell_4, \ell_5, \ell_6 \}$
	\item $\begin{aligned}[t]	\delta_P = \{ &(\ell_1,\ \texttt{x := 0},\ \ell_2), (\ell_2,\ \texttt{ x < 6},\ \ell_3), (\ell_3,\ \texttt{x := x + 2},\ \ell_2), \\ &(\ell_2,\ \texttt{!x < 6},\ \ell_4), (\ell_4,\ \texttt{x == 6},\ \ell_5), (\ell_4,\ \texttt{x != 6},\ \ell_6)\} \end{aligned}$
	\item  $\ell_{init} = \ell_1$
	\item $\ell_{err} = \ell_6$
\end{itemize}
And the following graphical representation: 

\begin{figure}[H]
	\centering
	\begin{tikzpicture}[%
		->,
		>=stealth', shorten >=1pt, auto,
		node distance=3cm, scale=1, 
		transform shape, align=center,    
		smallnode/.style={inner sep=1.4}
		initial text =]
		\node[state, initial above](1){$\ell_1$};
		
		\node[state] (2) [below of=1] {$\ell_2$};
		
		\node[state] (3) [left of=2] {$\ell_3$};
		
		\node[state] (4) [right of=2] {$\ell_4$};
		
		\node[state] (5) [below of=4] {$\ell_5$};
		
		\node[state] (6) [right of=4] {$\ell_6$};
		
		\path (1) edge node {\texttt{x := 0}} (2)
		(2) edge [bend right] node [above]{\texttt{x < 6}} (3)
		(3) edge [bend right] node [below]{\texttt{x := x + 2}} (2)
		(2) edge node {\texttt{!x < 6}} (4)
		(4) edge node {\texttt{x == 6}} (5)
		(4) edge node {\texttt{x != 6}} (6)
		;
	\end{tikzpicture}
	\captionof{figure}{Control-Flow Graph $G_{P_0}$ of Program $P_0$.}
	\label{fig:rect}
\end{figure}
Each program variable has a domain $D$ defining the set of all possible values.
\begin{mydef}
	A variable valuation of a program variable $v \in Var$ is a function $\rho: v \rightarrow D$  assigning it a value from its domain.
\end{mydef}
Assigning every program variable a valuation creates a program state.

%the n may be wrong, because traces can be infinite in theory?

\begin{mydef}
	Assume a program $P$ is defined over $n$ variables, a program state $\sigma$ is a function assigning each variable $v_i \in V$, \ $0 \leq i \leq n$ a variable valuation $\rho_i$. The set $S$ denotes the set of all program states.
\end{mydef}

Program statements can change the valuation of variables, transitioning one program state to another.

\begin{mydef}
	Each program statement $\texttt{s} \in Stmt$ defines a binary relation $\rho_s \subseteq S \times S$ over the set of program states $S$, called successor relation. Which is, given an interpretation function $\mathcal{I}$, inductively defined as \\
\end{mydef}
	 $ \rho_s =
\begin{cases}
	\{(\sigma, \sigma')\ |\ \mathcal{I}(\text{bexpr})(\sigma)\ =\ true\ \text{and}\ \sigma = \sigma'\} , & \text{if}\ s \equiv \text{\texttt{assume bexpr}} \\
	\{(\sigma, \sigma')\ |\ \sigma' = \sigma[x \mapsto \mathcal{I}(expr)(\sigma)]\} , & \text{if}\ s \equiv \texttt{x := expr} \\
	\{(\sigma, \sigma')\ |\ \exists \sigma''\ \text{where}\ (\sigma, \sigma'') \in \rho_{s_1}\ \text{and}\ (\sigma'', \sigma') \in \rho_{s_2} \}, & \text{if}\ s \equiv \texttt{$s_1;s_2$}
\end{cases}
$
\vspace{0.7cm} \\ 
Programs model multiple program statement sequences:
\begin{mydef}
Given a program $P = (Loc, \delta, \ell_{init}, \ell_{err})$ that is defined over a set of program statements $Stmt$, and its control-flow graph $G_P$, we call a sequence of program statements \\ $\tau: s_0, s_1, s_2, ..., s_n \in Stmt^*$ a program trace, if $\tau$ is the labeling of a path starting in $\ell_{init}$. Meaning, each for each $s_i$, $0 \leq i \leq n$, there is $(\ell_i, s_i, \ell_{i+1}) \in \delta$. The program trace is called error trace if $\ell_n = \ell_{err}$.  
\end{mydef}

For example, consider program $P_0$ from before. A possible error trace would be:
\begin{equation*}
	\tau_0:\ \ \texttt{x := 0},\ \texttt{x < 6},\ \texttt{x := x + 2},\ \texttt{!x < 6},\ \texttt{x != 6}
\end{equation*}
Graphically represented as:

\begin{figure}[H]
	\centering
	\begin{tikzpicture}[%
		->,
		>=stealth', shorten >=1pt, auto,
		node distance=2.5cm, scale=1, 
		transform shape, align=center,    
		smallnode/.style={inner sep=2}
		initial text =]
		
		\node[state] (2) {$\ell_1$};
		
		\node[state] (3) [right of=2] {$\ell_2$};
		
		\node[state] (4) [right of=3] {$\ell_3$};
		
		\node[state] (5) [right=3cm of 4] {$\ell_2$};
		
		\node[state] (6) [right of=5] {$\ell_4$};
		
		\node[state] (7) [right of=6] {$\ell_6$};
		
		
		\path (2) edge node {\texttt{x := 0}} (3); 
		\path (3) edge node {\texttt{x < 6}} (4); 
		\path (4) edge node {\texttt{x := x + 2}} (5); 
		\path (5) edge node {\texttt{!x < 6}} (6);
		\path (6) edge node {\texttt{ x != 6}} (7); 
		;
	\end{tikzpicture}
\end{figure}

Because a program trace is rather abstract, the question arises whether the program can actually take the defined sequence of statements.

\begin{mydef}
	Given a program $P = (Loc, \delta, \ell_{init}, \ell_{err})$ and a program trace $\tau:\ s_0, s_1, s_2, ..., s_n$ of $P$, a sequence of program states $\pi:\ \sigma_0, \sigma_1, \sigma_2,..., \sigma_n$ is a program execution of $\tau$, if $(\sigma_i, \sigma_{i+1}) \in \rho _{s_i}$ for $0 \leq i \leq n$. \\
	Trace $\tau$ is called feasible if there is a corresponding program execution, otherwise it is infeasible.
\end{mydef}

$\tau_0$ from before is infeasible, because for $\texttt{assume !x < 6}$, the interpretation function $\mathcal{I}(\texttt{!x < 6})(x = 2)$ evaluates to false and is, with that, not part of the successor relation.
%Are there annotations of the nodes correct?
\begin{figure}[H]
	\centering
	\begin{tikzpicture}[%
		->,
		>=stealth', shorten >=1pt, auto,
		node distance=2.5cm, scale=1, 
		transform shape, align=center,    
		smallnode/.style={inner sep=2}
		initial text =]
		
		\node[state, label=above:{$\top$}] (2) {$\ell_1$};
		
		\node[state, label=above:{$x = 0$}] (3) [right of=2] {$\ell_2$};
		
		\node[state, label=above: {$x = 0$}] (4) [right of=3] {$\ell_3$};
		
		\node[state, label=above:{$x = 2$}] (5) [right=3cm of 4] {$\ell_2$};
		
		\node[state, label=above:{$\bot$}] (6) [right of=5] {$\ell_4$};
		
		\node[state, label=above:{$\bot$}] (7) [right of=6] {$\ell_6$};
		
		
		\path (2) edge node {\texttt{x := 0}} (3); 
		\path (3) edge node {\texttt{x < 6}} (4); 
		\path (4) edge node {\texttt{x := x + 2}} (5); 
		\path (5) edge node {\texttt{!x < 6}} (6);
		\path (6) edge node {\texttt{ x != 6}} (7); 
		;
	\end{tikzpicture}
\end{figure}

Program trace $\tau_0$ will therefore never occur in $P_0$. \\

\begin{mydef}
	Given a program trace $\tau: s_0, s_1, ..., s_n$ a sequence of program states $\pi: \sigma_0, \sigma_1, ..., \sigma_n$ is called an inductive sequence of program states for $\tau$ if the strongest postcondition of every program state $\sigma_i$ under the program statement $st_{i+1}$ implies the follow up state $\sigma_{i+1}$. 
	\begin{equation*}
			sp(\sigma_i, st_{i+1}) \implies \sigma_{i+1}
	\end{equation*}
\end{mydef}
The sequence of program states $\pi_0: true, x = 0, x = 0, x = 2, false, false$ is an inductive sequence of program states for $\tau_0$

\begin{mydef}
	An inductive sequence  of program states $\pi: \sigma_0, \sigma_1,..., \sigma_n$ of program trace $\tau$ is a proof of infeasibility for $\tau$ if $\sigma_0 = true$ and $\sigma_n = false$.
\end{mydef}
A proof of infeasibility refutes the possibility of a program trace being taken in the program's execution.\\
To show that there are no feasible error traces, trace abstraction iteratively creates an automaton $A_D$, defined on the alphabet of programs statements $\Sigma$, with language $\mathcal{L}(A_D)$. In each iteration $A_D$ gets refined using an infeasibility proof, extending the language: $\mathcal{L}(A_D) = \mathcal{L}(A_D) \cup \pi$.
If the language $\mathcal{L}(A_D)$ subsumes the language of the original program $\mathcal{L_P}$, then the program is proven to uphold the safety property.
If there is a feasible error trace then a counterexample is found, proving that the program does not fulfill the safety property.

\jw{TODO}

\begin{equation}
	x_0 = 5 \land x_0 \leq 3
\end{equation}
Which is unsatisfiable. A derived interpolant is $x_0 = 5$ \\
	Interpolanten -> Sequenzinterpolanten \cite{10.1007/11691372_33}-> Interpolantenautomat

\vspace*{2cm}

\subsection{Interpolating Trace Abstraction}

By combining infeasibility proofs and interpolant computation we can define trace abstraction as a technique for proving or disproving safety for a given program by checking for reachability of an error location.
Given a program $P = (Loc, \delta, \ell_{init}, \ell_{err})$ and its control-flow graph $G_P$, to check reachability of error location $\ell_{err}$, use trace abstraction with interpolation according to the following paradigm \cite{10.1007/978-3-642-03237-0_7}: \\
\begin{itemize}
	\item[1.] Search $G_P$ for a program trace that starts at $\ell_{init}$ and ends in $\ell_{err}$:
	\begin{figure}[H]
		\centering
		\begin{tikzpicture}[%
			->,
			>=stealth', shorten >=1pt, auto,
			node distance=2.5cm, scale=1, 
			transform shape, align=center,    
			smallnode/.style={inner sep=2}
			initial text =]
			
			\node[state](1){$\ell_{init}$};
			
			\node[state] (2) [right of=1] {$\ell_1$};
			
			\node[] (3) [right of=2] {$\cdots$};
			
			\node[state] (4) [right of=3] {$\ell_{n-1}$};
			
			\node[state] (5) [right of=4] {$\ell_{err}$};
			
			
			\path (1) edge node {\texttt{$s_0$}} (2); 
			\path (2) edge node {$s_1$} (3); 
			\path (3) edge node {$s_{n-1}$} (4);
			\path (4) edge node {$s_n$} (5); 
			;
		\end{tikzpicture}
	\end{figure}
		\item[2.] Prove feasibility. \\ In case that the trace is proven feasible, the program is incorrect, if the trace is infeasible construct an infeasibility proof.
		\item[3.] Use the infeasibility proof to calculate interpolants.
		\item[4.] Construct an automaton, $\mathcal{A}_i$, from the interpolants.
		\item[5.] If the language $\mathcal{L(A_P)}$, that is recognized by the program's control-flow graph, is a subset of the union of languages recognized by the constructed automata: $\mathcal{L(A_P)} \subseteq \mathcal{L(A}_1) \cup ... \cup \mathcal{L(A}_i)$ then the program is correct, else start again at step 1.
	\end{itemize}
	The interpolants generated in step 3 serve to generalize the infeasibility proof to exclude other possible error traces. However, the interpolants are not guaranteed to be general enough to exclude a large number of error traces. Which poses a problem for loops. In the following we introduce a way to exclude a large number of traces going through a loop.


\section{Loop Acceleration}
Programs generally contain loops, like \texttt{while} or \texttt{for} loops. These can create up to infinitely many distinct program traces. Trace abstraction has then, in the worst case when the generated Craig interpolants are only general enough to disprove one trace, refute every program trace generated by the loop. A remedy to that is the computation of a loop acceleration in form of the reflexive transitive closure. This chapter introduces loops as traces and relations, and how a loop relation can be used to compute a reflexive transitive closure. \\

Assume we are given the following program $P_1$: \\
\begin{figure}[H]
	\begin{align*}
		&\texttt{1: int x := 0}; \\
		&\texttt{2: int y := 1}; \\
		&\texttt{3: while x <= 50:} \\
		&\texttt{4: \hspace*{2em} x := x + 1;} \\
		&\texttt{5: \hspace*{2em} y := y + 2;} \\
		&\texttt{6: end while} \\
		&\texttt{7: assert y == 103;}
	\end{align*}
	\captionof{figure}{Example Program $P_1$.}
	\label{fig:square}
\end{figure}
With its control-flow graph $G_{P_1}$: \\
\begin{figure}[H]
	\centering
	\begin{tikzpicture}[%
		->,
		>=stealth', shorten >=1pt, auto,
		node distance=3cm, scale=1, 
		transform shape, align=center,    
		smallnode/.style={inner sep=1.4}
		initial text =]
		
		\node[state, initial above](1){$\ell_1$};
		
		\node[state] (2) [below of=1] {$\ell_2$};
		
		\node[state] (7) [below of=2] {$\ell_3$};
		
		\node[state] (3) [left of=7] {$\ell_4$};
		
		\node[state] (8) [below of=3] {$\ell_5$};
		
		\node[state] (4) [right of=7] {$\ell_6$};
		
		\node[state] (5) [below of=4] {$\ell_7$};
		
		\node[state] (6) [right of=4] {$\ell_8$};
		
		\path (1) edge node {\texttt{x := 0}} (2)
		(2) edge node {\texttt{y := 1}} (7)
		(7) edge node[above] {\texttt{x <= 20}} (3)
		(3) edge node[left=0.25cm] {\texttt{x := x + 1}} (8)
		(8) edge[bend right] node[right=0.25cm] {\texttt{y := y + 2}} (7)
		(7) edge node {\texttt{!x <= 50}} (4)
		(4) edge node {\texttt{y == 103}} (5)
		(4) edge node {\texttt{y != 103}} (6)
		;
	\end{tikzpicture}
	\captionof{figure}{Control-Flow Graph $G_{P_1}$ of Program $P_1$.}
	\label{fig:rect}
\end{figure}
Assume trace abstraction generates program trace $\tau_1$:
\begin{figure}[H]
	\centering
	\begin{tikzpicture}[%
		->,
		>=stealth', shorten >=1pt, auto,
		node distance=3.25cm, scale=1, 
		transform shape, align=center,    
		smallnode/.style={inner sep=2}
		initial text =]
		
		\node[state](1){$\ell_1$};
		\node[state] (2) [right of=1] {$\ell_2$};
		\node[state] (3) [right of=2] {$\ell_3$};
		\node[state] (4) [right of=3] {$\ell_4$};
		\node[state] (5) [right of=4] {$\ell_5$};
		\node[state] (6) [below of=1] {$\ell_3$};
		\node[state] (7) [right of=6] {$\ell_4$};
		\node[state] (8) [right of=7] {$\ell_5$};
		\node[state] (9) [right of=8] {$\ell_3$};
		\node[state] (10) [right of=9] {$\ell_4$};
		\node[state] (11) [below of=6] {$\ell_5$};
		\node[state] (12) [right of=11] {$\ell_3$};
		\node[state] (13) [right of=12] {$\ell_6$};
		\node[state] (14) [right of=13] {$\ell_7$};
		
		
		\path (1) edge node {\texttt{x := 0}} (2);
		\path (2) edge node {\texttt{y := 1}} (3);
		\path (3) edge node {\texttt{x <= 50}} (4);
		\path (4) edge node {\texttt{x := x + 1}} (5);
		\path (5) edge node {\texttt{y := y + 2}} (6);
		\path (6) edge node[below] {\texttt{x <= 50}} (7);
		\path (7) edge node[below] {\texttt{x := x + 1}} (8);
		\path (8) edge node[below] {\texttt{y := y + 2}} (9);
		\path (9) edge node {\texttt{x <= 50}} (10);
		\path (10) edge node {\texttt{x := x + 1}} (11);
		\path (11) edge node[below] {\texttt{y := y + 2}} (12);
		\path (12) edge node[below] {\texttt{!x <= 50}} (13);
		\path (13) edge node[below] {\texttt{y != 103}} (14);
		;
	\end{tikzpicture}
	\captionof{figure}{Program Trace $\tau_1$ of $P_1$}
\end{figure}
It is noticeable that program location $\ell_3$ appears repeatedly in the trace indicating that it is an entry point for a loop. From this entry point, the so called loop head, one can extract a loop trace. 
\begin{mydef}
		Given a program $P = (Loc, \delta, \ell_{init}, \ell_{err})$ and a program trace $\tau: s_0, s_1, \ldots, s_n$. $\tau$ is a loop trace, if $(\ell_0, s_0, \ell_1) \in \delta\ $ and $(\ell_{n-1}, s_n, \ell_0) \in \delta $. $\ell_0$ is called the loop head.
\end{mydef}

In $\tau_1$ there are three loop traces of varying length:  \\

\begin{figure}[H]
	\centering
		\begin{tikzpicture}[%
			->,
			>=stealth', shorten >=1pt, auto,
			node distance=3.25cm, scale=1, 
			transform shape, align=center,    
			smallnode/.style={inner sep=2}
			initial text =]
			
			\node[state] (2) [] {$\ell_3$};
			\node[state] (3) [right of=2] {$\ell_4$};
			\node[state] (4) [right of=3] {$\ell_5$};
			\node[state] (5) [right of=4] {$\ell_3$};
			
			\path (2) edge node {\texttt{x <= 50}} (3);
			\path (3) edge node {\texttt{x := x + 1}} (4);
			\path (4) edge node {\texttt{y := y + 2}} (5);
			;
		\end{tikzpicture}
		\captionof{figure}{Loop Trace $\tau_{L_1}$ modelling one iteration of the loop.}
\end{figure}

\begin{figure}[H]
	\centering
	\begin{tikzpicture}[%
		->,
		>=stealth', shorten >=1pt, auto,
		node distance=3.25cm, scale=1, 
		transform shape, align=center,    
		smallnode/.style={inner sep=2}
		initial text =]
		
		\node[state] (2) [] {$\ell_3$};
		\node[state] (3) [right of=2] {$\ell_4$};
		\node[state] (4) [right of=3] {$\ell_5$};
		\node[state] (5) [right of=4] {$\ell_3$};
		\node[state] (6) [below of=2] {$\ell_4$};
		\node[state] (7) [right of=6] {$\ell_5$};
		\node[state] (8) [right of=7] {$\ell_3$};
		
		\path (2) edge node {\texttt{x <= 50}} (3);
		\path (3) edge node {\texttt{x := x + 1}} (4);
		\path (4) edge node {\texttt{y := y + 2}} (5);
		\path (5) edge node {\texttt{x <= 50}} (6);
		\path (6) edge node[below] {\texttt{x := x + 1}} (7);
		\path (7) edge node[below] {\texttt{y := y + 2}} (8);
		;
	\end{tikzpicture}
	\captionof{figure}{Loop Trace $\tau_{L_2}$ Modelling two Iterations of the Loop.}
\end{figure}

\begin{figure}[H]
	\centering
	\begin{tikzpicture}[%
		->,
		>=stealth', shorten >=1pt, auto,
		node distance=3.25cm, scale=1, 
		transform shape, align=center,    
		smallnode/.style={inner sep=2}
		initial text =]
		
		\node[state] (2) [] {$\ell_3$};
		\node[state] (3) [right of=2] {$\ell_4$};
		\node[state] (4) [right of=3] {$\ell_5$};
		\node[state] (5) [right of=4] {$\ell_3$};
		\node[state] (6) [below of=2] {$\ell_4$};
		\node[state] (7) [right of=6] {$\ell_5$};
		\node[state] (8) [right of=7] {$\ell_3$};
		\node[state] (9) [right of=8] {$\ell_4$};
		\node[state] (10) [below of=6] {$\ell_5$};
		\node[state] (11) [right of=10] {$\ell_3$};
		
		\path (2) edge node {\texttt{x <= 50}} (3);
		\path (3) edge node {\texttt{x := x + 1}} (4);
		\path (4) edge node {\texttt{y := y + 2}} (5);
		\path (5) edge node {\texttt{x <= 50}} (6);
		\path (6) edge node[below] {\texttt{x := x + 1}} (7);
		\path (7) edge node[below] {\texttt{y := y + 2}} (8);
		\path (8) edge node {\texttt{x <= 50}} (9);
		\path (9) edge node {\texttt{x := x + 1}} (10);
		\path (10) edge node[below] {\texttt{y := y + 2}} (11);
		;
	\end{tikzpicture}
	\captionof{figure}{Loop Trace $\tau_{L_3}$ Modelling three Iterations of the Loop.}
\end{figure}
These three looping traces represent the same loop but with a different number of iterations.
\begin{mydef}
	A minimal loop trace $\tau_{min}$ is a loop trace $s_0, s_1, \ldots, s_n$, where the loop head $\ell_L$ only appears as first and last location. It represents one iteration of the loop.
\end{mydef}

In this example $\tau_{L_1}$ is the minimal loop trace. \\

With the minimal loop trace it is possible to formulate the overall effect the loop has on the program state.
\begin{mydef}
	The loop relation $\psi_L$ describes the effect a loop has on the program state.
	Given the minimal looping trace $\tau_{min}: s_0, s_1, \ldots, s_{n}$ the loop relation can be constructed by using the composition of all program statements $s_i$.
	\begin{equation*}
		\psi_L = s_0; s_1; \ldots; s_n
	\end{equation*}

\end{mydef}
For $\tau_{L_1}$ we get the following loop relation:
\begin{align*}
 	\psi_{L_1}&:	\texttt{x <= 50; x := x + 1; y := y + 2} \ \\
 	&= \{(\sigma, \sigma')\ |\ \sigma[x] \leq 50 \land \sigma'[x] = \sigma[x] + 1 \land \sigma'[y] = \sigma'[y] + 2 \}
\end{align*}
The loop relation can appear in a program trace an infinite amount of times, leading to infinitely many program traces. It is however possible to contain every looping trace in a single relation, the so called reflexive transitive closure. \\

Analogous to the composition of program statements, as detailed in definition 4, it is possible to concatenate relations.
\begin{mydef}
	The concatenation of two relations $\psi_1$ and $\psi_2$, that are defined over the set of program states, is defined as:
	\begin{equation*}
		\psi_1 \circ \psi_2 = \{(\sigma, \sigma')\ |\ \exists \sigma''\ \text{where}\ (\sigma, \sigma'') \in \psi_1\ \text{and}\ (\sigma'', \sigma') \in \psi_2 \}
	\end{equation*}
\end{mydef}

This concatenation operator is utilized in the reflexive transitive closure's definition.

\begin{mydef}
	Given loop relation $\psi_L$, the reflexive transitive closure $\psi_L^*$ is a relation that includes every possible loop trace. It is inductively defined as follows:
	\begin{itemize}
		\item $\psi^*_L = \bigvee_{i=0}^\infty \psi^i_L$
		\item $\psi^i_L = $
		$\begin{cases}
			& \varepsilon \hspace{1.5cm} \text{if}\ i = 0 \\
			& \psi \circ \psi^{i - 1} \hspace{0.4cm}\text{otherwise}
		\end{cases}$
	\end{itemize}
With identity relation $\varepsilon: \{(\sigma, \sigma')\ |\ \sigma = \sigma'\}$
\end{mydef}
There are multiple techniques of computing the reflexive transitive closure of a loop relation, this paper will focus on an algorithm based on ultimately periodic relations, as detailed and implemented in previous works\cite{JillThesis}. \\

For our example loop relation  	$\psi_{L_1}$ we calculate the reflexive transitive closure:
\begin{align*}
	\psi^*_{L_1}: \{(\sigma, \sigma') \  |\ &((\sigma'[x] \leq \sigma[x]\ \lor\ \sigma'[x] \leq 51)\ \land\ (\sigma[x] \leq \sigma'[x])\ \land\ \\ & (\sigma'[y] = \sigma[y] - 2 \cdot \sigma[x] + 2\cdot \sigma'[x]))\ \lor\ (\sigma'[x] = \sigma[x]\ \land\ \sigma'[y] = \sigma[y]) \}
\end{align*}

This reflexive transitive closure will be used in the following chapter to compute more general interpolants to be used in trace abstraction.

\section{Accelerated Interpolation}
This section will introduce the technique of combining loop acceleration and interpolating trace abstraction to check safety of a program. We will show multiple types of loops, beginning with loops without branching, continuing with loops with branching, and finishing with nested loops. \\

\subsection{Loops Without Branching}

Reconsider program $P_1$ from the previous chapter. To prove its safety trace abstraction finds error trace $\tau_1$, as before, we detect a loop with  $\ell_3$ and minimal loop trace $\tau_{L_1}$ from which we construct the loop relation $\psi_{L_1}$. It is evident that there is only one path through the loop, resulting in a loop without branching. We consequently compute the loop acceleration $\psi^*_{L_1}$. \par

The loop acceleration contains every looping trace, meaning it is possible to replace the whole loop in the error trace by that acceleration, modelling a relation consisting of every loop trace.

\begin{mydef}
	Given an error trace $\tau$ containing loop with loop head $\ell_L$. A meta trace $\bar{\tau}$ is derived from $\tau$ by replacing the sequence of program statements between the first and last occurrence of $\ell_L$ by a single edge labelled with the loop acceleration. The last occurrence of $\ell_L$ is replaced by a new loop exit $\ell_L'$.
\end{mydef}

Error trace $\tau_1$ creates meta trace $\bar{\tau_1}$:


\begin{figure}[H]
\begin{tikzpicture}[%
->,
>=stealth', shorten >=1pt, auto,
node distance=2.5cm, scale=1, 
transform shape, align=center,    
smallnode/.style={inner sep=1.4}
initial text =]

\node[state](1){$\ell_1$};

\node[state] (2) [right of=1] {$\ell_2$};

\node[state] (3) [right of=2] {$\ell_3$};

\node[state] (4) [right of=3] {$\ell_3'$};

\node[state] (5) [right of=4, xshift=0.5cm] {$\ell_6$};

\node[state] (6) [right of=5, xshift=0.5cm] {$\ell_7$};

\path (1) edge node {\texttt{x := 0}} (2); \\
\path (2) edge node {\texttt{y := 1}} (3); \\
\path (3) edge node {$\psi^*_{L_1}$} (4);\\
\path (4) edge node[] {\texttt{!x <= 50}} (5); \\
\path (5) edge node {\texttt{y != 103}} (6); \\
;
\end{tikzpicture}
	\captionof{figure}{Meta trace $\bar{\tau_1}$ Generated from $\tau_1$ Using $\psi^*_{L_1}$.}
\end{figure}

We can now analyse this meta trace for feasibility using an SMT-solver such as SMTInterpol\cite{Zitat03} or z3\cite{z3}. We get the following labelling:

\begin{figure}[H]
		\begin{tikzpicture}[%
		->,
		>=stealth', shorten >=1pt, auto,
		node distance=2.5cm, scale=1, 
		transform shape, align=center,    
		smallnode/.style={inner sep=1.4}
		initial text =]
		
		\node[state, label=above:{$\top$}](1){$\ell_1$};
		
		\node[state, label=above:{$x = 0$}] (2) [right of=1] {$\ell_2$};
		
		\node[state, label={above:$\begin{aligned}
				&y = 1 \\ \land\ &x = 0
			\end{aligned}$}] (3) [right of=2] {$\ell_3$};
		
		\node[state, label={[align=center]above:
			$\begin{aligned}
				((102 < y\ &\lor \ 2x + 1 \leq y)\ \\ \land\ (y &\leq 103)) \\ \lor \ x &= 0\ \\
			\end{aligned}$}] (4) [right of=3, xshift=0.5cm] {$\ell_3'$};
		
		\node[state, label=above:{$y = 103$}] (5) [right of=4, xshift=0.5cm] {$\ell_6$};
		
		\node[state, label=above:{$\bot$}] (6) [right of=5, xshift=0.5cm] {$\ell_7$};
		
		\path (1) edge node {\texttt{x := 0}} (2); \\
		\path (2) edge node {\texttt{y := 1}} (3); \\
		\path (3) edge node {$\psi^*_{L_1}$} (4);\\
		\path (4) edge node[] {\texttt{!x <= 50}} (5); \\
		\path (5) edge node {\texttt{y != 103}} (6); \\
		;
	\end{tikzpicture}
	\captionof{figure}{Meta Trace $\bar{\tau_1}$ Generated from $\tau_1$ and $\psi^*_{L_1}$ and Its Infeasibility Proof.}
\end{figure}
With interpolant sequence:
\begin{equation*}
	 I_{\bar{\tau_1}}: \{ \top,\ x = 0, y = 1 \land x = 0,\ ((102 < y\ \lor \ 2x + 1 \leq y)\ \land\ (y \leq 103))\ \lor\ x = 0,\ y = 103,\ \bot \}
\end{equation*}
We cannot, however, use $I_{\bar{\tau_1}}$ to disprove $\tau_1$ as we need an interpolant for each location in the original trace. To remedy this, we derive an inductive interpolant sequence $I_{\tau_1}$ by applying the post operator. \\

\begin{mydef}
	Given an error trace $\tau: s_0, s_1, \ldots , s_i, \ldots , s_j, \ldots , s_n$ containing a loop spanning from $s_1$ to $s_j$ with loop head $\ell_i$, loop relation $\psi_L$, and corresponding loop acceleration $\psi^*_{L}$ and its derived meta trace $\bar{\tau}: s_0, s_1, \ldots, \psi^*_{L}, \ldots , s_n$ with an infeasibility proof consisting of interpolant sequence: $I_{\bar{\tau}}: \{\top, I_1, I_2, \ldots , I_{\psi^*_{L}}, \ldots , I_{n-1}, \bot \}$. 
	To construct an inductive proof of infeasibility for $\tau$ we need inductive interpolants for the loop statements $s_i, \ldots , s_j$ that were replaced by $\psi^*_{L}$. \\
	Firstly, compute post($I_{\psi^*_L}$, $\psi^*_L$) as the loop entry interpolant $I_{\ell_L}$. From there keep applying the post operator with the previous location's interplant and the following program statement. \\
	We get inductive interpolant sequence
	\begin{equation*}
		I_\tau: \{\top,I_1,I_2, \ldots ,\ \underbrace{post(I_{\psi^*_L}, \psi^*_L)}_{I_{i}},\ \ \underbrace{post(I_{i}, s_i)}_{I_{i+1}},\ \ldots ,\ \underbrace{post(I_{j-1}, s_j)}_{I_{j}},I_{j+1}, \ldots ,I_{n-1}, \bot \}
	\end{equation*}
	Which can now be used by trace abstraction to refine the interpolant automaton.
\end{mydef}

Using definition 15 we can now compute the missing interpolants for example program trace $\tau_1$:

\begin{figure}[H]
	\centering
	\begin{tikzpicture}[%
		->,
		>=stealth', shorten >=1pt, auto,
		node distance=3.25cm, scale=1, 
		transform shape, align=center,    
		smallnode/.style={inner sep=2}
		initial text =]
		
		\node[state, label=above:{$\top$}](1){$\ell_1$};
		\node[state, label=above:{$x = 0$}] (2) [right of=1] {$\ell_2$};
		\node[state, label=above:{$\begin{aligned}
				&y = 2 \cdot x + 1\\&\land x \leq 51
			\end{aligned}$}] (3) [right of=2] {$\ell_3$};
		\node[state, label=above:{$\begin{aligned}
				&y = 2 \cdot x + 1\\&\land x \leq 50
			\end{aligned}$}] (4) [right of=3] {$\ell_4$};
		\node[state, label=above:{$\begin{aligned}
				&y + 1 = 2 \cdot x\\&\land x \leq 51
			\end{aligned}$}] (5) [right of=4] {$\ell_5$};
		\node[state, label=above:{$\begin{aligned}
				&y = 2 \cdot x + 1\\&\land x \leq 51
			\end{aligned}$}] (6) [below of=1] {$\ell_3$};
		\node[state, label=below:{$\begin{aligned}
				&y = 2 \cdot x + 1\\&\land x \leq 50
			\end{aligned}$}] (7) [right of=6] {$\ell_4$};
		\node[state, label={[xshift=0.7cm]above:{$\begin{aligned}
				&y + 1 = 2 \cdot x\\&\land x \leq 51
			\end{aligned}$}}] (8) [right of=7] {$\ell_5$};
		\node[state, label={[xshift=0cm]above:{$\begin{aligned}
				&y = 2 \cdot x + 1\\&\land x \leq 51
				\end{aligned}$}}] (9) [right of=8] {$\ell_3$};
		\node[state, label={[xshift=0cm]above:{$\begin{aligned}
				&y = 2 \cdot x + 1\\&\land x \leq 50
				\end{aligned}$}}] (10) [right of=9] {$\ell_4$};
		\node[state, label=above:{$\begin{aligned}
				&y + 1 = 2 \cdot x\\&\land x \leq 51
			\end{aligned}$}] (11) [below of=6] {$\ell_5$};
		\node[state, label={[xshift=0cm]below:{$\begin{aligned}
				&y = 2 \cdot x + 1\\&\land x \leq 51
				\end{aligned}$}}] (12) [right of=11] {$\ell_3$};
		\node[state, label={[xshift=0cm]below:{$\begin{aligned}
					&y = 2 \cdot x + 1\\&\land x \leq 51\ \land\ x > 50 \\
					& \equiv y = 103
				\end{aligned}$}}] (13) [right of=12] {$\ell_6$};
		\node[state, label=above:{$\bot$}] (14) [right of=13] {$\ell_7$};
		
		
		\path (1) edge node {\texttt{x := 0}} (2);
		\path (2) edge node {\texttt{y := 1}} (3);
		\path (3) edge node {\texttt{x <= 50}} (4);
		\path (4) edge node {\texttt{x := x + 1}} (5);
		\path (5) edge node[left, xshift=-.7cm] {\texttt{y := y + 2}} (6);
		\path (6) edge node[below] {\texttt{x <= 50}} (7);
		\path (7) edge node[below] {\texttt{x := x + 1}} (8);
		\path (8) edge node[below] {\texttt{y := y + 2}} (9);
		\path (9) edge node {\texttt{x <= 50}} (10);
		\path (10) edge node {\texttt{x := x + 1}} (11);
		\path (11) edge node[below] {\texttt{y := y + 2}} (12);
		\path (12) edge node[below] {\texttt{!x <= 50}} (13);
		\path (13) edge node[below] {\texttt{y != 103}} (14);
		;
	\end{tikzpicture}
	\captionof{figure}{Program Trace $\tau_1$ of $P_1$ with Inductive Infeasibility Proof.}
\end{figure}

\subsection{Loops With Branching}
\jw{TODO: Text and defs}
Assume we are given the following program $P_2$: \\
\begin{figure}[H]
	\begin{align*}
		&\texttt{1: int x := 0}; \\
		&\texttt{2: while x <= 10:} \\
		&\texttt{3: \hspace*{2em} if x - 4 > 1:} \\
		&\texttt{4: \hspace*{4em} x := x + 1;} \\
		&\texttt{5: \hspace*{2em} else:} \\
		&\texttt{6: \hspace*{4em} x := x + 2;} \\
		&\texttt{5: \hspace*{2em} x := x + 2} \\
		&\texttt{7: end while} \\
		&\texttt{8: assert x == 11;}
	\end{align*}
	\captionof{figure}{Example Program $P_1$.}
	\label{fig:square}
\end{figure}
With its control-flow graph $G_{P_2}$: \\
\begin{figure}[H]
	\centering
	\begin{tikzpicture}[%
		->,
		>=stealth', shorten >=1pt, auto,
		node distance=3cm, scale=1, 
		transform shape, align=center,    
		smallnode/.style={inner sep=1.4}
		initial text =]
		
		\node[state, initial above](1){$\ell_1$};
		
		\node[state] (2) [below of=1] {$\ell_2$};
		
		\node[state] (3) [left of=2, xshift=-3cm] {$\ell_3$};
		
		\node[state] (8) [below right of=3] {$\ell_4$};
		
		\node[state] (9) [below left of=3] {$\ell_5$};
		
		\node[state] (10) [below left of=8] {$\ell_6$};
		
		\node[state] (11) [right of=2] {$\ell_7$};
		\node[state] (12) [below right of=11] {$\ell_8$};
		\node[state] (13) [below left of=11] {$\ell_{9}$};
		
		\path (1) edge node {\texttt{x := 0}} (2)

		(2) edge node[above] {\texttt{x <= 10}} (3)
		(3) edge node[right=0.25cm] {\texttt{x - 4 > 1}} (8)
		(3) edge node[left=0.25cm] {\texttt{!x - 4 > 1}} (9)
		
		(8) edge node[right=0.25cm] {\texttt{x := x + 1}} (10)
		(9) edge node[left=0.25cm] {\texttt{x := x + 2}} (10)
		(10) edge [bend angle=55, bend right] node[right=0.25cm] {\texttt{x := x + 2}} (2)
		
		(2) edge node[above] {\texttt{!x <= 10}} (11)
		(11) edge node[above right] {\texttt{x == 11}} (12)
		(11) edge node[below right] {\texttt{x != 11}} (13)
		;
	\end{tikzpicture}
	\captionof{figure}{Control-Flow Graph $G_{P_2}$ of Program $P_2$.}
	\label{fig:rect}
\end{figure}


\begin{figure}[H]
	\centering
	\begin{tikzpicture}[%
		->,
		>=stealth', shorten >=1pt, auto,
		node distance=3.25cm, scale=1, 
		transform shape, align=center,    
		smallnode/.style={inner sep=2}
		initial text =]
		
		\node[state](1){$\ell_1$};
		\node[state] (2) [right of=1] {$\ell_2$};
		\node[state] (3) [right of=2] {$\ell_3$};
		\node[state] (4) [right of=3] {$\ell_5$};
		\node[state] (5) [right of=4] {$\ell_6$};
		\node[state] (6) [below of=1] {$\ell_2$};
		\node[state] (7) [right of=6] {$\ell_3$};
		\node[state] (8) [right of=7] {$\ell_5$};
		\node[state] (9) [right of=8] {$\ell_6$};
		\node[state] (10) [right of=9] {$\ell_2$};
		\node[state] (11) [below of=6] {$\ell_3$};
		\node[state] (12) [right of=11] {$\ell_4$};
		\node[state] (13) [right of=12] {$\ell_6$};
		\node[state] (14) [right of=13] {$\ell_2$};
		\node[state] (15) [right of=14] {$\ell_3$};
		\node[state] (16) [below of=11] {$\ell_4$};
		\node[state] (17) [right of=16] {$\ell_6$};
		\node[state] (18) [right of=17] {$\ell_2$};
		\node[state] (19) [right of=18] {$\ell_7$};
		\node[state] (20) [right of=19] {$\ell_9$};
		
		
		\path (1) edge node {\texttt{x := 0}} (2);
		\path (2) edge node {\texttt{x <= 10}} (3);
		\path (3) edge node {\texttt{!x - 4 > 1}} (4);
		\path (4) edge node {\texttt{x := x + 2}} (5);
		\path (5) edge node {\texttt{x := x + 2}} (6);
		\path (6) edge node[below] {\texttt{x <= 10}} (7);
		\path (7) edge node[below] {\texttt{!x - 4 > 1}} (8);
		\path (8) edge node[below] {\texttt{x := x + 2}} (9);
		\path (9) edge node {\texttt{x := x + 2}} (10);
		\path (10) edge node {\texttt{x <= 10}} (11);
		\path (11) edge node[below] {\texttt{x - 4 > 1}} (12);
		\path (12) edge node[below] {\texttt{x := x + 1}} (13);
		\path (13) edge node[below] {\texttt{x := x + 2}} (14);
		\path (14) edge node {\texttt{x <= 10}} (15);
		\path (15) edge node {\texttt{x - 4 > 1}} (16);
		\path (16) edge node[below] {\texttt{x := x + 1}} (17);
		\path (17) edge node[below] {\texttt{x := x + 2}} (18);
		\path (18) edge node[below] {\texttt{!x <= 10}} (19);
		\path (19) edge node[below] {\texttt{x != 11}} (20);
		;
	\end{tikzpicture}
	\captionof{figure}{Program Trace $\tau_2$ of $P_2$}
\end{figure}

Loop relations and accelerations:
\begin{align*}
 	\psi_{L_{1,1}} &:\ \texttt{x <= 10; !x - 4 > 1; x := x + 2;\ x := x + 2} \\
	&= \{(\sigma, \sigma')\ |\ \sigma[x] \leq 5 \land  \sigma'[x] = \sigma[x] + 4 \} \\
	\psi_{L_{1,1}}^* &: \{(\sigma, \sigma')\ |\ (\sigma'[x] \leq 9 \lor \sigma[x]' \leq x)\ \land\ 3\cdot \sigma'[x] + \sigma[x]\ (mod\ -4) = 0\ \land\ \sigma[x] \leq \sigma'[x] \} \\ \\
	\psi_{L_{1,2}}&:\ \texttt{x <= 10; x - 4 > 1; x := x + 1; x := x + 2} \\
	&= \{(\sigma, \sigma')\ |\ \sigma[x] \leq 10 \land \sigma[x] > 5 \land  \sigma'[x] = \sigma[x] + 3 \} \\
	\psi_{L_{1,2}}^* &: \{ (\sigma, \sigma')\ | \ \sigma'[x] = \sigma[x]\ \lor\ \sigma[x] \leq \sigma'[x]\ \land\ \sigma'[x] + 2 \cdot \sigma[x]\ (mod\ 3) = 0 \\
	&\hspace*{1.5cm} \land\ 3 < \sigma[x]\ \land\ \sigma'[x] \leq 13 \} \\
\end{align*}

Branching Meta-Trace:
\begin{figure}[H]
	\begin{tikzpicture}[%
		->,
		>=stealth', shorten >=1pt, auto,
		node distance=2.5cm, scale=1, 
		transform shape, align=center,    
		smallnode/.style={inner sep=1.4}
		initial text =]
		
		\node[state](1){$\ell_1$};
		
		\node[state] (2) [right of=1] {$\ell_2$};
		
		\node[state] (3) [right of=2, xshift=-0.55cm] {$\ell_2'$};
		
		\node[state] (4) [right of=3, xshift=-0.55cm] {$\ell_2$};
		
		\node[state] (5) [right of=4, xshift=-0.55cm] {$\ell_2'$};
		
		\node[state] (6) [right of=5, xshift=0.4cm] {$\ell_7$};
		
		\node[state] (7) [right of=6, xshift=0.4cm] {$\ell_9$};
		
		\path (1) edge node {\texttt{x := 0}} (2);
		\path (2) edge node {$\psi^*_{L_{1,1}}$} (3);
		\path (3) edge node {$\varepsilon$} (4);
		\path (4) edge node {$\psi^*_{L_{1,2}}$} (5);
		\path (5) edge node {\texttt{!x <= 10}} (6);
		\path (6) edge node {\texttt{x != 11}} (7);
		;
	\end{tikzpicture}
	\captionof{figure}{Meta Trace $\bar{\tau_2}$ Generated from $\tau_1$ Using a Loop with Branching.}
\end{figure}

\begin{figure}[H]
	\begin{tikzpicture}[%
		->,
		>=stealth', shorten >=1pt, auto,
		node distance=2.5cm, scale=1, 
		transform shape, align=center,    
		smallnode/.style={inner sep=1.4}
		initial text =]
		
		\node[state, label=above:{$\top$}](1){$\ell_1$};
		
		\node[state, label=above:{$x = 0$}] (2) [right of=1] {$\ell_2$};
		
		\node[state, xshift=-0.55cm, label={above:$\begin{aligned}
				&(5 \cdot x \leq 56 \\ &\lor x \leq 4) \\ &\land x \leq 9
			\end{aligned}$}] (3) [right of=2] {$\ell_2'$};
		
		\node[state, xshift=-0.55cm, label={above:$\begin{aligned}
				&(5 \cdot x \leq 56 \\ &\lor x \leq 4) \\ &\land x \leq 9
			\end{aligned}$}] (4) [right of=3] {$\ell_2$};
		
		\node[state,xshift=-0.55cm,  label=above:{$ x \leq 11$}] (5) [right of=4] {$\ell_2'$};
		
		\node[state, xshift=0.4cm, label=above:{$x = 11$}] (6) [right of=5] {$\ell_7$};
		
		\node[state, xshift=0.4cm, label=above:{$\bot$}] (7) [right of=6] {$\ell_9$};
		
		\path (1) edge node {\texttt{x := 0}} (2); \\
		\path (2) edge node {$\psi_{L_{1,1}}$} (3); \\
		\path (3) edge node {$\varepsilon$} (4);\\
		\path (4) edge node {$\psi_{L_{1,2}}$} (5); \\
		\path (5) edge node {\texttt{!x <= 10}} (6); \\
		\path (6) edge node {\texttt{x != 11}} (7); \\
		;
	\end{tikzpicture}
	\captionof{figure}{Meta Trace $\bar{\tau_1}$ Generated from $\tau_1$ and $\psi^*_{L_1}$ and Its Infeasibility Proof.}
\end{figure}
\subsection{Nested Loops}
Assume we are given the following program $P_3$: \\
\begin{figure}[H]
	\begin{align*}
		&\texttt{1: int x := 0}; \\
		&\texttt{2: int y := 0}; \\
		&\texttt{3: while x < 5:} \\
		&\texttt{4: \hspace*{2em} x := x + 1;} \\
		&\texttt{5: \hspace*{2em} while y < 4:} \\
		&\texttt{6: \hspace*{4em} y := y + 2;} \\
		&\texttt{7: \hspace*{2em} end while} \\
		&\texttt{8: end while} \\
		&\texttt{9: assert y == 10;}
	\end{align*}
	\captionof{figure}{Example Program $P_3$.}
	\label{fig:square}
\end{figure}

\begin{figure}[H]
	\centering
	\begin{tikzpicture}[%
		->,
		>=stealth', shorten >=1pt, auto,
		node distance=3cm, scale=1, 
		transform shape, align=center,    
		smallnode/.style={inner sep=1.4}
		initial text =]
		
		\node[state, initial above](1){$\ell_1$};
		
		\node[state] (2) [below of=1] {$\ell_2$};
		
		\node[state] (3) [below of=2] {$\ell_3$};
		
		\node[state] (4) [left of=3] {$\ell_4$};
		
		\node[state] (5) [left of=4] {$\ell_5$};
		
		\node[state] (6) [below of=5] {$\ell_6$};
		
		\node[state] (8) [right of=3] {$\ell_8$};
		
		\node[state] (9) [below of=8] {$\ell_9$};
		
		\node[state] (10) [right of=8] {$\ell_{10}$};
		
		\path (1) edge node {\texttt{x := 0}} (2)
		(2) edge node {\texttt{y := 0}} (3)
		(3) edge node {\texttt{x < 5}} (4)
		(4) edge[] node {\texttt{x := x + 1}} (5)
		(5) edge[bend right] node[left] {\texttt{y < 4}} (6)
		(6) edge[bend right] node[right] {\texttt{y := y + 2}} (5)
		(5) edge[bend left] node[] {\texttt{!y < 10}} (3)
		(3) edge[] node[] {\texttt{!x < 5}} (8)
		(8) edge[] node[] {\texttt{y != 10}} (9)
		(8) edge[] node[] {\texttt{y == 10}} (10)
		;
	\end{tikzpicture}
	\captionof{figure}{Control-Flow Graph $G_{P_3}$ of Program $P_3$.}
	\label{fig:rect}
\end{figure}

\pagebreak
\addcontentsline{toc}{chapter}{Bibliography}
\bibliographystyle{plain}
\bibliography{bib}

	
\end{document}