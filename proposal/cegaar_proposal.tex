\documentclass{article}

\usepackage[utf8]{inputenc}
\usepackage{xspace}
\usepackage{tabu}
\usepackage[%
  hyperindex,%
  plainpages=false,%
  pdfusetitle]{hyperref}
\usepackage[all]{hypcap}
\usepackage{cite}
\usepackage{booktabs}
\usepackage{url}
\usepackage{listings}
\usepackage{enumitem}
\usepackage{amsthm}
\usepackage{amsmath}
\usepackage{tikz}
\usetikzlibrary{positioning,shapes.geometric, arrows,automata, decorations.pathreplacing, calc}
\usepackage{pgf}
\usepackage{slantsc}
\usepackage{geometry}
\usepackage{amssymb}
\usepackage{subcaption}
\usepackage{float}
\usepackage{pgf}
\usepackage{slashbox}
\usepackage{pgfgantt}
\usepackage{wrapfig}
\usepackage{pdflscape}



\usepackage[%disable,%
  colorinlistoftodos,%
  color=cyan!50!white,%
  bordercolor=cyan!50!black]{todonotes}

%%%%%%%%%%%% Colors 
%% a somewhat friendly scheme for 5 different colors 
\definecolor{g1}		{RGB}{215,25,28} % a kind of red
\definecolor{g2}		{RGB}{253,174,97} % a kind of orange
\definecolor{g3}		{RGB}{255,255,191} % a kind of yellow
\definecolor{g4}		{RGB}{171,217,233} % a kind of light blue 
\definecolor{g5}		{RGB}{44,123,182} % a kind of dark blue 

\definecolor{gr1}		{RGB}{250, 250, 250}
\definecolor{gr2}		{RGB}{229, 229, 229} % some grey

% color of interpolants
\definecolor{grey}{RGB}{200,200,200}

%color for pictures
\colorlet{outlineblue}		{g5}
\colorlet{fillblue}			{g4}
\colorlet{darkback}			{gr2}
\colorlet{lightback}		{gr1}
\colorlet{stmtcolor}		{gr2} %default statement color
\colorlet{subgraphcolor}	{g3} %default statement color


%%%%%%%%%%%% Setup
\newtheorem{name}{Printed output}
\newtheorem{mydef}{Definition}

\hypersetup{
colorlinks=true,        % false: boxed links; true: colored links
linkcolor=g1,        % color of internal links
citecolor=g1,        % color of links to bibliography
filecolor=g1,        % color of file links
urlcolor=g1          % color of external links
}


\lstdefinestyle{boogie}{
  belowcaptionskip=1\baselineskip,
  breaklines=true,
  xleftmargin=\parindent,
  showstringspaces=false,
  basicstyle=\footnotesize\ttfamily,
  numbers=left,
  xleftmargin=.6cm
}

\lstset{escapechar=@,style=boogie}

%%%%%%%%%%%% Comments
\newif\iffinal
%\finaltrue % comment out to remove comments 
 
\iffinal
\newcommand\mycom[1]{}
\else
\newcommand\mycom[1]{#1}
\overfullrule=1mm
\fi
\setlength\parindent{0pt}

\newcommand{\WidestEntry}{$\psi_{global}$}%
\newcommand{\SetToWidest}[1]{\makebox[\widthof{\WidestEntry}]{$#1$}}%

\newcommand{\jw}[1]{\mycom{\todo[color=blue!40,inline]{\small JW: #1}}}
\newcommand{\dd}[1]{\mycom{\todo[color=orange!40,inline]{\small DD: #1}}}
\newcommand{\ts}[1]{\mycom{\todo[color=green!40,inline]{\small TS: #1}}}


\newcommand{\all}[1]{\mycom{\todo[color=green!40,inline]{\small #1}}}
\newcommand{\meta}[1]{\mycom{\todo[color=blue!10,inline,caption={Beschreibung},nolist]{\setlist{nolistsep}\small #1}}}
\newcommand{\xxx}{\mycom{\stfootcol{Placeholder}{blue!20}\xspace}}
\newcommand{\cn}{\mycom{\stfootcol{Cite}{blue!20}\xspace}}


\begin{document}
	\newcommand{\HorizontalLine}{\rule{\linewidth}{0.3mm}}
	
		\begin{center}
		{\scshape\Large Master Project \par}
		\vspace{1.5cm}
		{\huge\bfseries Trace Abstraction with Accelerated Interpolants \par}
		{\Huge\itshape Proposal \par}
		\vspace{1cm}
		{\large \scshape Jonas Werner\par}
		\vspace{0.5cm}
		{\today \vspace{2cm}} 
		
		\end{center}

\section{Introduction}
Assume we want to verify whether a program fulfills a given safety property. The program's control-flow graph represents every possible trace that can be taken by a program execution. Executions that violate the safety property end in a so called error location. Error traces are traces in the control-flow graph that start in the initial node and end in the error location. The goal is to check if there are feasible error traces.
For this purpose one can apply an automata-based instance of the CEGAR scheme, called trace abstraction, on the program's control-flow graph. \\

Trace abstraction aims at constructing automata \cite{10.1007/978-3-642-39799-8_2} from infeasible error traces. When the language recognized by the program's control-flow graph is a subset of these automata, it means, every possible error trace, and by that execution which ends in an error location, is infeasible. Proving that the program fulfills the safety property. If there is a single feasible error trace, then the program violates the safety property.\\
\par 
If an error trace is infeasible there is an infeasibility proof. Using this proof one can construct an automaton, which excludes the original error trace from the control-flow graph. However, this excludes only one error trace, making this approach not very efficient. To exclude more than one error trace, one can try to compute a generalization of the infeasibility proof. \par

A common strategy is \cite{10.1007/978-3-642-03237-0_7} to calculate a generalization using Craig interpolation \cite{craig_1957}, where an SMT-solver computes a sequence of interpolants from an infeasibility proof. \par 
But, it is not guaranteed that these interpolants are more general. This issue is most notably in program loops. Assume that the given program contains a loop with guard $x < 5000$, $x$ being an integer variable. It is possible that the computed interpolant sequence does not exclude every loop iteration, leading to the need of disproving every of the 5000 possible error traces individually. \\ \\
A solution for this problem is accelerating the loop, meaning computing its transitive closure, and calculating interpolants on that. \par This project aims at implementing exactly that. 
The goal is to combine interpolation and loop acceleration on the basis of the work of Hojjat et al \cite{10.1007/978-3-642-33386-6_16} in the software analysis framework \\ Ultimate \cite{Zitat02}. \par
The remainder of this proposal is structured as follows. Chapter 2 will give an overview of needed background information, like a more detailed look at trace abstraction, loop acceleration, and the combination of interpolation and acceleration. Chapter 3 will detail the approach this project will take to implement accelerated interpolation in Ultimate, and finally an outline of the project's deliverables and schedule.

\section{Background}
This project aims at combining loop acceleration and interpolant calculation based on the findings of Hojjat et al. \cite{10.1007/978-3-642-33386-6_16}, however, instead of utilizing a CEGAR-scheme with predicate abstraction, it will be implemented as an automata-based CEGAR-scheme, called trace abstraction. \par
This section will introduce the basic ideas behind trace abstraction, loop acceleration, and finally accelerated interpolants.

\subsection{Interpolating Trace Abstraction}
A program $\texttt{P}$ can be represented by a control-flow graph.
\begin{mydef}
	A program $\texttt{P}$'s control-flow graph $G_p = (Loc, \Delta, \ell_{0}, \ell_{Err})$ is a tuple, where
	 \begin{itemize}
		\item $Loc$ is a set of program locations
		\item $\Delta$ is a set of triples $(\ell_i, stm_i, \ell_{i + 1})$ representing program transitions with $\ell_i, \ell_{i + 1} \in Loc$ and program statement $stm_i$
		\item $\ell_{0}$ represents the initial location
		\item $\ell_{Err} \subseteq Loc$ a set of error locations
	\end{itemize}
\end{mydef}
It is possible to interpret the control-flow graph as a nondeterministic automaton. \\ 

For example, consider the following program $\texttt{P}$ and its corresponding control-flow graph $G_P$:

\begin{minipage}{.3\textwidth}
	\centering
	\begin{align*}
	&\texttt{1: int x := 0}; \\
	&\texttt{2: while x <= 4;} \\
	&\texttt{3: \hspace*{2em} x := x + 3;} \\
	&\texttt{4: end while} \\
	&\texttt{5: assert x == 6}
	\end{align*}
	\captionof{figure}{Program $\texttt{P}$}
	\label{fig:square}
\end{minipage}%
\hfill
\begin{minipage}{.6\textwidth}
	\centering
	\begin{tikzpicture}[%
	->,
	>=stealth', shorten >=1pt, auto,
	node distance=2.5cm, scale=1, 
	transform shape, align=center,    
	smallnode/.style={inner sep=1.4}
	initial text =]
	\node[state, initial above](1){$\ell_0$};
	
	\node[state] (2) [below of=1] {$\ell_1$};
	
	\node[state] (3) [left of=2] {$\ell_2$};
	
	\node[state] (4) [right of=2] {$\ell_3$};
	
	\node[state] (5) [below of=4] {$\ell_E$};
	
	\path (1) edge node {$\texttt{x := 0}$} (2)
	(2) edge [bend right] node [above]{$\texttt{x <= 4}$} (3)
	(3) edge [bend right] node [below]{$\texttt{x := x + 3}$} (2)
	(2) edge node {$\texttt{!x <= 4}$} (4)
	(4) edge node {$\texttt{x != 6}$} (5)
	;
	\end{tikzpicture}
	\captionof{figure}{$G$}
	\label{fig:rect}
\end{minipage}%
\\ \vspace*{2em}

Through a control-flow graph there are so called program traces characterizing transitions from one program state to another, starting at the initial location. \\

\begin{mydef}
	A program trace is a sequence of tuples $(\ell_0, \varphi_0, \ell_1), (\ell_1, \varphi_1, \ell_2), ... (\ell_{n - 1}, \varphi_{n - 1}, \ell_n),$ where each tuple $(\ell_i, \varphi_i, \ell_{i + 1})$ consists of
	\begin{itemize}
		\item $\ell_i$ the source program location
		\item $\varphi_i$ a first order formula in single static assignment form defining the changesto the program variables during a state transition, called the transition formula
		\item  $\ell_{i+1}$ the target location after the transition
	\end{itemize}
\end{mydef}

A possible 

 To check reachability of an error location $\ell_E \in \ell_{Err}$, use trace abstraction with interpolation according to the following paradigm \cite{10.1007/978-3-642-03237-0_7}:
\begin{itemize}
	\item[1.] Search the program's control-flow graph for a trace that starts at $\ell_0$ and ends in $\ell_E$:
	\begin{equation*}
			 \ell_{0} \xrightarrow{\text{$stm_1$}} \ell_1 \xrightarrow{\text{$stm_2$}} \cdots \xrightarrow{\text{$stm_{n-1}$}} \ell_{n-1} \xrightarrow{\text{$stm_{n}$}} \ell_{E}
	\end{equation*}
	With each $(\ell_i, stm_i, \ell_{i + 1}) \in \Delta$.
	\item[2.] Prove feasibility. \\ In case that the trace is proven feasible, the program is incorrect, if the trace is infeasible construct an infeasibility proof.
	\item[3.] Use the infeasibility proof to calculate interpolants.
	\item[4.] Construct an automaton, $\mathcal{A}_i$, from the interpolants.
	\item[5.] If the language $\mathcal{L(A_P)}$, that is recognized by the program's control-flow graph, is a subset of the union of languages recognized by the constructed automata: $\mathcal{L(A_P)} \subseteq \mathcal{L(A}_1) \cup ... \cup \mathcal{L(A}_i)$ then the program is correct, else start again at step 1.
\end{itemize}
The interpolants generated in step 3 serve to generalize the infeasibility proof to exclude other possible error traces. However, the interpolants are not guaranteed to be general enough to exclude infinitely many error traces. Which poses a problem for loops. In the following we introduce a way to exclude an infinite number of traces going through a loop.


\section{Accelerated Interpolation}
This section will introduce the technique of combining loop acceleration and interpolation computation for usage in trace abstraction. \\\\


To check the reachability of $\ell_E$ we use the trace abstraction paradigm: \\
Firstly, we need a possible error trace $\tau_0$. \\ \\

\begin{figure}[H]
\begin{tikzpicture}[%
->,
>=stealth', shorten >=1pt, auto,
node distance=2.5cm, scale=1, 
transform shape, align=center,    
smallnode/.style={inner sep=1.4}
initial text =]

\node[state](1){$\ell_0$};

\node[state] (2) [right of=1] {$\ell_1$};

\node[state] (3) [right of=2] {$\ell_2$};

\node[state] (4) [right of=3, xshift=1cm] {$\ell_1$};

\node[state] (5) [right of=4] {$\ell_3$};

\node[state] (6) [right of=5] {$\ell_E$};

\path (1) edge node {$x_0 = 0$} (2); \\
\path (2) edge node {$x_0 \leq 4$} (3); \\
\path (3) edge node {$x_1 = x_0 + 3$} (4);\\
\path (4) edge node {$x_1 > 4$} (5); \\
\path (5) edge node {$x_1 \neq 6$} (6); \\
;
\end{tikzpicture}
	\captionof{figure}{$\tau_0$}
\end{figure}

\subsection{Loop Acceleration}


\subsection{Meta-Traces}
The transitive closure of a loop contains every trace going through it. To make use of loop acceleration, we need to pull apart the looping location $\ell_2$ by introducing so called meta-transitions of the form:
\begin{equation*}
\overset{stm_3\ \circ \ stm_4}{\overset{\curvearrowright}{\ell_2}} \Rightarrow \hspace*{1cm} \ell_2' \xrightarrow{\text{$(stm_3 \circ stm_4)^*$}} \ell_2''
\end{equation*}
Where $(stm_3 \circ stm_4)^*$ symbolizes the calculated transitive closure of the loop. \par
Using meta-transitions, we can transform our trace scheme into a meta-trace:
\begin{equation*}
\bar{\tau}: \ell_0 \xrightarrow{\text{$stm_1$}} \ell_1 \xrightarrow{\text{$stm_2$}} \ell_2' \xrightarrow{\text{$(stm_3 \circ stm_4)^*$}} \ell_2'' \xrightarrow{\text{$stm_5$}} \ell_E
\end{equation*}
The feasibility of the meta-trace is the same as the trace scheme before: \\
\begin{itemize}
	\item  If it is feasible then the original trace is feasible, making the program incorrect
	\item If it is infeasible, we can compute an interpolant sequence: \\
	\begin{equation*}
	I_{\bar{\tau}}: \langle \top, I_1, I_2', I_2'', \bot  \rangle
	\end{equation*}
\end{itemize}

To guarantee inductiveness to the loop, we have to compute the strongest post of each interpolant, that coincide with a location in the loop, and its transitive closure:
\begin{equation*}
I_{\bar{\tau}}^{post}: \langle \top, I_1, post(I_2', stm_3 \circ stm_4), \bot  \rangle
\end{equation*}
This sequence is now general enough for the trace scheme to exclude the loop.



\section{Approach}
We want to implement accelerated interpolation into the software verification framework Ultimate \cite{Zitat02}. Ultimate consists of multiple different libraries that can be executed in serial to form so called toolchains. There are already toolchains implementing trace abstraction as described above. This project extends these toolchains with accelerated interpolation by creating a new interpolating trace checking library. \par 
This library will be used for step 3 of the trace abstraction paradigm. For that it has to fulfill the following specifications: 
\begin{itemize}
	\item Computing loop accelerations of octagonal relations using ultimately periodic relations as shown by Bozga et al \cite{10.1007/978-3-642-14295-6_23}. And being able to implement further acceleration techniques in the future.
	\item Construct trace schemes from given error traces and extend them using loop acceleration to meta schemes as presented by Hojjat et al \cite{10.1007/978-3-642-33386-6_16}.
	\item Check meta schemes' feasibility
	\item Construct interpolants for infeasible meta schemes to be used in trace abstraction.
\end{itemize}
This project is finished when this library has been implemented, tested, and evaluated by using it on several verification tasks. \\ \\
Master projects award 16 ECTS points which translates to 16 weeks of work. Resulting in the following approach:
\begin{itemize}
	\item[1.] \textsl{Understanding the Matter:} \\
	Grasp how the combination of loop acceleration and interpolation works. \\
	Moreover, there have been previous projects that implemented the basis \cite{ClausThesis} and the ultimately periodic acceleration scheme itself \cite{JillThesis} in Ultimate. It is important to understand those implementations before trying to adapt them for the usage of accelerated interpolants. \\ \\
	\textsl{Duration in weeks:} 1 \\\\
	\textsl{Result}: Being able to use the previous work and having an understanding on how to implement an accelerated interpolation library.
	\item[2.] \textsl{Implement Trace Scheme Transformation:} \\
	The accelerated interpolation library should be able to transform a given error trace into a trace scheme. \\\\
	\textsl{Duration in weeks:} 4 \\\\
	\textsl{Result}: Given an error trace, accelerated interpolation is able to transform it into a trace scheme.
	
	\item[3.] \textsl{Implement Own Version of Loop Acceleration:} \\
	Implement a way of using Fast Acceleration of Ultimately Periodic Relations in the accelerated interpolation library. \\ \\
	\textsl{Duration in weeks:} 4 \\ \\
	\textsl{Result}: Accelerated interpolation is able to construct the transitive closure of loops to use for interpolant generation. And is able to be extended to use other acceleration methods in the future.
	
	\item[4.] \textsl{Combining Trace Schemes and Loop Acceleration:} \\
	Accelerated interpolation can construct meta-schemes from trace schemes by inserting the transitive closure of the loop, prove its feasibility, and furthermore, is able to generate interpolants from the meta-scheme. \\ \\
	\textsl{Duration in weeks:} 3 \\\\
	\textsl{Result}: Accelerated interpolation returns a sequence of interpolants for use in the trace abstraction paradigm, or returns that the program is incorrect, depending on the feasibility of an error trace.
	
	\item[5.] \textsl{Evaluating the Library:} \\
	Search for possible bugs and compare it to other trace checkers, such as PDR. \\\\
	\textsl{Duration in weeks:} 2 \\\\
	\textsl{Result}: A bugfree accelerated interpolation library and data comparing it to other trace checkers.
	
	\item[6.] \textsl{Writing a Documentation about the Project:} \\
	Write in more detail than this proposal how accelerated interpolation works, and what new knowledge we discovered during this project \\\\
	\textsl{Duration in weeks:} 2 \\\\
	\textsl{Result}: A written documentation about this project.
\end{itemize}
\begin{landscape}
	\section{Schedule}
	\pagenumbering{gobble}
	This schedule illustrates the approach and shows the sequence of the individual tasks. \\ \\
	\hspace*{-4cm}
	\begin{tikzpicture}
	\begin{ganttchart}[hgrid, vgrid, x unit=1.25cm, bar height=.6,, title height=1, canvas/.append style={alias=frame}
	]{1}{16}
	
	\gantttitle{Time in weeks}{16} \ganttnewline
	
	\gantttitlelist{1,...,16}{1} \\
	
	
	\ganttbar{1. Understanding}{1}{1}
	
	% A NEW BAR FOR EACH WITH INLINE
	
	\ganttbar[inline]{}{1}{1} \ \\
	
	\ganttbar{2. Trace Scheme}{2}{5}
	\ganttbar[inline]{}{2}{5} \ \\
	
	\ganttbar{3. Acceleration}{6}{9}
	\ganttbar[inline]{}{6}{9} \ \\
	
	\ganttbar{4. Combining}{10}{12}
	\ganttbar[inline]{}{10}{12} \ \\
	
	\ganttbar{5. Evaluating}{13}{14}
	\ganttbar[inline]{}{13}{14} \ \\
	
	\ganttbar{6. Writing}{15}{16}
	\ganttbar[inline]{}{15}{16}
	\end{ganttchart}
	\useasboundingbox (frame.south west) rectangle (frame.north east);
	\end{tikzpicture}
	
\end{landscape}
\pagebreak
\addcontentsline{toc}{chapter}{Bibliography}
\bibliographystyle{plain}
\bibliography{bib}

	
\end{document}